\section{Étude de la TMA (UO 2.2.1 à 2.2.3)}
	\subsection{Compréhension du besoin et démarche proposée}
	\subsubsection{Enjeux}
	Les enjeux de l’étude de la TMA sont très importants. En effet cela est déterminant pour le MOA car elle va permettre de connaître la faisabilité des tâches à accomplir. Il est question d’étudier s’il est possible d’effectuer des évolutions fonctionnelles dans l’environnement \correlyce{}. De plus, il faut évaluer les contraintes techniques matérielles associées à ces évolutions (ajouts, modifications ou remplacements).
	
	Cette étude sera guidée par le niveau d’exigence du MOA. Il pourra varier et dépendra des possibilités et des contraintes mises en avant pendant cette étape.
	
	Cependant, cette UO n’est pas obligatoire et ne sera faite que si le MOA estime qu’il est nécessaire de faire une étude au préalable.
	
	\subsubsection{Méthodologie}
	Dans un premier temps il faudra déterminer le niveau d’étude à effectuer. Trois niveaux d’étude sont possibles et elles sont échelonnées en fonction des contraintes liées à l’existant, aux paramétrages et aux principes.
	
	La première étape de l’étude sera de prendre en compte l’expression des besoins initiaux du MOA. Il faudra également prendre en considération l’environnement technique de \correlyce{} pour pouvoir mettre en évidence les contraintes techniques. En addition à cela, il faudra prendre en considération les exigences fonctionnelles et non fonctionnelles (en termes de reprise de données, de migration, d'interopérabilité, de performances, de qualité de service… ). Pour finir, les normes, les standards et les contraintes générales d’architecture devront également faire partie de l’étude.
	
	Nous allons alors procéder à des réunions où les parties prenantes et l’équipe de développement travailleront en collaboration afin de mettre en évidence les \textit{users stories} à réaliser ainsi que les contraintes qui leurs sont associées. Ces réunions auront lieu fréquemment en fonction de la priorité des \textit{stories}, ceci dans l’optique de rester dans la démarche agile de Kanban.
	
	La seconde étape consistera à faire l’étude à proprement parler. L’étude sera désignée dans le workflow Kanban comme une story. Nos ingénieurs seront affectés à cette tâche, ils seront chargés de réfléchir à la meilleure solution possible et devront fournir les livrables associés. 
	
	La liste des travaux à effectuer pour une étude est énoncée ci-dessous et cette analyse de l’évolution fonctionnelle devra respecter les exigences du MOA ainsi que la SDET:
	
	\begin{itemize}
		\item Vérifier que l’évolution technique peut fonctionner en adéquation avec l’architecture (logicielle et matérielle) déjà en place. Dans le cas contraire, il devra être proposé au MOA des recommandations détaillées pour pouvoir s’adapter au problème.
		\item Identifier les opérations nécessaires à la configuration et au paramétrage des composants logiciels et identifier les opérations de développement spécifique.
		\item Identifier le niveau d’interopérabilité avec les autres modules fonctionnels ainsi que l’interfaçage avec les annuaires et les bases de données.
		\item Identification de la charte graphique attente du MOA.			
		\item Identification du niveau d’homogénéité ergonomique nécessaire avec les modules  internes et externes de \correlyce{}.
		\item Proposer des moyens de contournement ou d’adaptation si des limitations ou des difficultés sont détectées pour réaliser l’implémentation de l’évolution.
		\item Définir les opérations de migration et de reprise de données si cela est nécessaire.
		\item Déterminer la nature des tests à réaliser pour garantir la robustesse opérationnelle.
		\item Élaborer un protocole de mise en oeuvre complète de l’évolution.
		\item Estimer les charges nécessaires pour l’implémentation de l’évolution et quantifier les UO requises.
		\end{itemize}
		
	Toutes ces tâches devront être alors restituées de manière orale à l’occasion des réunions planifiées par le MOA. De plus, il devra être transmis pour chacune des réunions les supports de présentations et les rapports de l’étude en cours.
	
	\subsubsection{Points forts}
	Nos ingénieurs possèdent déjà une très bonne expérience en ce qui concerne l’étude de TMA. John Du Bois et Pauline Marechal ont déjà participé à plusieurs TMA et sont habitués à travailler dans la même équipe. De plus, ils peuvent profiter de l’expertise des autres collaborateurs de notre entreprise. En effet, nos valeurs agiles et notre politique axée sur le partage des connaissances permettent de faciliter l’entraide entre nos équipes.
	
	\subsection{Compétences de l'équipe dédiée spécifiquement à l'UO}
	Pour assurer une prestation de qualité, nous avons mis sur ce projet une équipe d’ingénieurs aux compétences complémentaires. John Du Bois étant un ingénieur senior ayant acquis beaucoup d’expériences dans le cadre des TMA, il dispose des compétences nécessaires pour guider le projet. Il sera épaulé par Pauline Marechal qui est une collaboratrice en qui nous avons extrêmement confiance et qui a déjà fait ses preuves.
	
	Table \ref{table:competences1}, nous avons résumé les compétences clefs en relation avec l’étude de TMA de nos deux ingénieurs:
	
	\begin{table}[H]
			\centering
	\begin{tabular}{|p{3cm}|p{1.8cm}|c|p{3cm}|p{3cm}|p{3cm}|}
		\hline
		\textbf{Nom} & \textbf{Gestion d'équipe} & \textbf{Kanban} & \textbf{Analyse interopérabilité} & \textbf{Robustesse opérationnelle} & \textbf{Année d'expérience}\\
		\hline
		John du \bsc{Bois} & \checkmark & \checkmark &\checkmark  & & 15 ans\\
		\hline
		Pauline \bsc{Marechal} & &\checkmark  & &\checkmark  & 4 ans\\
		\hline
	\end{tabular}
	\caption{Compétences de l'équipe pour l'étude de TMA}
	\label{table:competences1}
	\end{table}
	
		\begin{table}[H]
			\centering
\begin{tabular}{|p{3cm}|p{3cm}|p{3cm}|p{3cm}|}
	\hline
\multirow{2}{*}{\textbf{Nom}}&\multicolumn{3}{c|}{\textbf{Nombre de jour par étude}}\\ 
\cline{2-4}
& Simple & Moyenne & Complexe \\ 
	\hline John du \bsc{Bois} & 2J & 3J & 10J \\ 
	\hline Pauline \bsc{Marechal} & 3J & 5J & 15J \\ 
	\hline 
\end{tabular} 
			\caption{Répartition des ressources pour l'étude de TMA}
			\label{table:ressources1}
		\end{table}
		\subsubsection{Cas similaires}
		L’année dernière nous avons eu l’occasion de collaborer avec la Mairie de Toulouse dans le cadre de l’étude de la maintenance d’un projet concernant la gestion des ressources affectant l’entretien des voiries. Nos deux ingénieurs étaient en charge du projet et nous avons utilisé Kanban pour le mener à bien.
		
		Un deuxième projet sur lequel nous avons été amenés à travailler consistait à faire la maintenance évolutive d’une application de gestion des services proposés par la Mairie de la ville d’Albi. Notre équipe a pu assurer à la fois la maintenance de l’application à travers la correction de bogues, ainsi que l’étude et l'implantation de nouvelles fonctionnalités de manière continue en fonction des idées introduites par le client. La mise en place d’une telle maintenance a nécessité une évaluation de la charge de travail pour les différents membres de l’équipe en se basant sur des estimations de complexité ainsi qu’un souci de réactivité durant la phase de réalisation.
		
	\subsection{Moyens techniques}
	Nous allons tout d’abord mettre en place des métriques afin d’analyser la qualité du code et sa couverture de tests. Ces métriques seront calculées à l’aide de PMD, CheckStyle et Emma. 
	En ce qui concerne la gestion des livrables et autres éléments de documentation, nous allons utiliser Git pour le versionnement, GitHub pour la gestion des taches et la Suite Libre Office pour la documentation utilisateur. Un plan de gestion de configuration sera rédigé par nos deux ingénieurs et permettra de réaliser un suivi des études effectuées par l’équipe. 
	