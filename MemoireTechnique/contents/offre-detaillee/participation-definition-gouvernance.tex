	\section{Participation à la définition d’une gouvernance logicielle (UO 2.6)}
	\subsection{Compréhension du besoin et démarche proposée}
	\subsubsection{Enjeux}
	Notre groupe devra être capable de participer à la définition du gouvernance logicielle en collaboration avec le prestataire du lot \no{}3.
	
	L’objectif de cette unité d'\oe{}uvre étant la collaboration entre notre groupe et le titulaire du lot \no{}3 afin d’arriver à un plan de gestion de configuration identique et pertinent pour ces différents lots.
	
	Pour cela, notre groupe devra participer à des réunions avec les autres parties responsables de la gouvernance logicielle afin d’arriver à des choix justes en ce qui concerne la gestion de configuration.  
	
	
	\subsubsection{Méthodologie}
	Un dispositif d’échange entre notre groupe et le titulaire du lot \no{}3 sera mis en place pour permettre notre entière collaboration avec ce groupe. Ce dispositif sera composé d’échanges téléphoniques, d’échange électroniques (mails, dépôts partagés) ainsi que de multiples réunions.
	
	
	\subsubsection{Points forts}
	Notre équipe dispose de plusieurs membres avec des connaissances non négligeables sur la gestion de configuration de logiciels et seront capables de fournir de multiples propositions dans le but d’obtenir une gouvernance logicielle détaillée et pertinente.
	
	\subsection{Compétences de l'équipe dédiée spécifiquement à l'UO}	
	
	Au sein de notre groupe, John Du Bois et Pauline Marechal possèdent de réelles compétences concernant la gestion de configuration d’un logiciel ce qui leur permet d’avoir un regard critique et constructif sur le sujet. John Du Bois est également qualifié en ce qui concerne l’architecture de logiciels. Enfin, l’ensemble des membres de notre équipe possède un sens du relationnel aiguisé. Ce qui est non négligeable lors de travaux en collaboration avec d’autres entités.
	
	\begin{table}[H]
		\centering
		\begin{tabular}{|c|c|c|c|}
			\hline
			\textbf{Nom} & \textbf{Gestion de configuration} & \textbf{Relationnel} & \textbf{Architecture}\\
			\hline
			John du \bsc{Bois} & \checkmark & \checkmark &\checkmark\\
			\hline
			Pauline \bsc{Marechal} & \checkmark  & \checkmark&\\
			\hline
			Rémy \bsc{Stukof} & &\checkmark  &\\
			\hline
		\end{tabular}
		\caption{Compétences de l'équipe pour la la gouvernance}
		\label{table:competences6}
	\end{table}
	
	
	\subsection{Moyens techniques}	
	Les documents produits au cours de la gouvernance seront gérés et historisés à l’aide de Git et GitHub. La suite Libre office permettra de créer des éléments de documentation.
	Des mails, téléphones, Zimbra et des salles de réunions seront mises à disposition pour optimiser la communication entre les équipes. Des réunions communes à l’ensemble des projets pourront être organisées par notre équipe afin de stimuler les échanges et permettre une cohérence conceptuelle et technique entre les différents lots et parties prenantes. 