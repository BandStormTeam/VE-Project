	\section{Prestation journalière à profil (UO 2.5.1 à 2.5.3)}
		\subsection{Compréhension du besoin et démarche proposée}
		\subsubsection{Enjeux}		
		Il est important pour la maîtrise d’ouvrage de pouvoir demander la mobilisation d’une ou plusieurs personnes de notre équipe sous la forme d’une prestation journalière. Lors de ces prestations, le ou les membres sollicités devront réaliser des tâches de développement ou de support sur l’environnement \correlyce{}.
		
		Les trois profils de prestataires suivants devront être disponibles; ingénieur expert, ingénieur et développeur pour permettre à la maîtrise d’ouvrage de sélectionner le ou les profils dont ils ont besoin pour cette prestation. 
		
		\subsubsection{Méthodologie}	
		Si le MOA demande la mobilisation d’une partie de l’équipe pour réaliser une prestation journalière, celle ci devra se rendre disponible pour effectuer la tâche indiquée. John Du Bois aura la responsabilité de gérer ce genre de cas de figure avec le MOA.
		
		\subsubsection{Points forts}
		Notre équipe est composée des différents profils nécessaires à cette unité d’oeuvre et nous disposons de membres qualifiés dont les compétences sont reconnues pour chaque niveau d’expertise.
		
		\subsection{Compétences de l'équipe dédiée spécifiquement à l'UO}	
		Ci-dessous nous avons résumé les compétences en relation avec les prestations journalières de notre équipe de développement:
		
		
		\subsection{Moyens techniques}
		Nous allons mettre en place un système de communication complet et efficace avec le MOA. Celui-ci devra être composé des éléments suivants:
		
		Des téléphones pour les membres de l’équipe.
		Des boites mails.
		
		John Du Bois devra utiliser les issues GitHub pour affecter les tâches aux membres de l’équipe concernés. 
		
		