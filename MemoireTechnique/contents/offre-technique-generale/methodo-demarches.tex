\section{Méthodologie et démarche}
\begin{exemple}
\textbf{Hypothèse} Plusieurs projets du client ont déjà été réalisés avec succès à l’aide de la méthode Scrum. Le client connaît donc bien cette méthode et à l’habitude de privilégier celle-ci sur les projets pouvant s’y prêter.
\end{exemple}

Les méthodes agiles ayant fait leurs preuves sur des projets de l’envergure de \correlyce{}, il nous semble opportun d’utiliser l’agilité pour la gestion de ce projet. Cette gestion de projet permettra de répondre de la manière la plus efficace et rapide à vos besoins. Cependant en raison des contraintes inhérentes au cadre des TMA, nous privilégions la méthode Kanban à la méthode Scrum. 

En effet, nous ne pouvons pas effectuer un travail de planification comme nécessaire dans le cadre de la planification d’un Sprint dans Scrum, du fait qu’on ne peut prévoir l’apparition de bogues. La méthode Kanban nous permettra d’ajouter au fur et à mesure les \textit{stories} à traiter en fonction de leur priorité. Ces priorités permettront de corriger les problèmes les plus urgents en premiers et ainsi d’assurer la meilleure fiabilité de la plateforme.

Généralement nous utilisons six colonnes afin de gérer le \textit{workflow} et garantir une prise en compte optimale des priorités et le bon avancement du projet : «\textit{backlog}», «à faire», «en analyse», «en développement», «à tester», «terminé». 
La colonne «\textit{Backlog}» répertorie l’ensemble des \textit{stories} et tâches à réaliser et la colonne “à faire” concerne les prochaines \textit{stories} à traiter. 
À ces colonnes (hormis «\textit{backlog}» et «terminé») nous affectons des limites en nombres, variables en fonction de la taille de l’équipe. Nous faisons attention à affecter un nombre relativement faible à la colonne «à tester», cela nous permet de corriger les éventuelles erreurs rapidement après le développement, le développeur ayant réalisé la \textit{story} a ainsi encore en mémoire son développement. 

Ce processus nous permet de gérer au mieux les risques, en cas de surgissement de bogues majeurs par exemple, comme le développement lié aux maintenances évolutives. Les limites des colonnes sont pour nous des variables d’ajustements. Par cette méthode nous mettons ainsi en valeur une excellente visibilité de l’avancement des \textit{stories} et un contrôle du \textit{workflow} nécessaire à tout appel d’offre public. 

Nous sommes ainsi en mesure de vous proposer à la fois une maîtrise du déroulement du projet et une flexibilité qui est bénéfique à sa réalisation. 

