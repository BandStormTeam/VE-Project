\section{Aptitudes face aux unités d'œuvres}
	Les principaux enjeux décelés dans le lot 2 portent sur la TMA de l’application \correlyce{}. Il est demandé par le biais d’une démarche agile de répondre aux unités d’oeuvre ci-dessous: 
	
	\begin{itemize}
		\item Étude de la TMA
		\item Effectuer la TMA
		\item Réversibilité du projet
		\item Dispositif d’assistance
		\item Participation à la définition d’une gouvernance logiciel
		\item Prestation journalière à profil 
	\end{itemize}

	Forts de notre longue expérience dans le développement d’application et dans la maintenance, nous sommes les candidats idéaux pour le projet \correlyce{}. Nous disposons d’experts maîtrisant les méthodes agiles. Nos équipes de développement sont équilibrées, présentant un expert et deux ingénieurs expérimentés au minimum, et un effectif suffisant pour pouvoir répondre à la montée en charge du projet. Cette composition permet à la fois à nos équipes de délivrer un travail de développement de qualité dans les temps impartis et de réaliser les études TMA grâce à l’expertise de ses membres. \\ Le recours aux logiciels libres et le développement sous licence libre sont partie intégrante de notre politique.
	
	Notre équipe de développement a acquis de l’expérience dans les tierces maintenances applicatives ainsi que dans l’étude et la réalisation de plusieurs plateformes de travail collaboratif pour les secteurs éducatifs et professionnels. Nous avons pu réaliser des plateformes pour diverses entreprises comme Continental, AXA et la Banque de Montréal pour des prestations entre 200 000 et 300 000\euro{}.
	
	Ces prestations ont fait l’objet d’enquêtes de qualité et de satisfaction disponibles sur notre site. Nos clients sur ces projets font aussi parties de nos références et sont prêts à être contactés concernant nos réalisations. Nous avons aussi eu l’occasion de réaliser un ENE pour le réseau éducatif Bruxellois en 2013 dont la courbe d’utilisation progresse de façon constante depuis plus d’un an. Outre ces différentes expériences au sujet des environnements numériques, notre groupe a également été en charge de TMA pour plusieurs entreprises et collectivités, dont Tisséo, L'Equipe.fr et l’office du tourisme de la ville d’Albi.\\
	Nos prestations sur ces TMA allant de 50000 à 350000\euro{} en fonction de la complexité et des ressources nécessaires pour les réaliser.
	
	Notre politique axée sur la qualité nous permet d’assurer les réversibilités dans les meilleures conditions. Nos projets sont soumis à une documentation rigoureuse. La mise en place d’un plan de configuration adapté au projet nous permet de garantir la bonne transmission de notre ingénierie et de notre système. C’est pourquoi nous sommes tout à fait à même de participer à la gouvernance logicielle dans le cadre de cette TMA.
	
	Afin de mieux accompagner nos clients, nous mettons en place un dispositif d’assistance pour chacun de nos projets.
	\newpage
	