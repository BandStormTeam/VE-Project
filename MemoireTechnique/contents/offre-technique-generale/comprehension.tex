\section{Compréhension du besoin}
Il y a un enjeu de taille à faire évoluer le numérique dans l’enseignement, afin d’améliorer ce système d’apprentissage.  
Ainsi, un Environnement Numérique Éducatif (ENE) a pour but d’unifier et de fournir un certain nombre de services aux lycéens de la région. Ces environnements ont pour but d’améliorer le lien entre élèves et professeurs (Messagerie interne, agenda, …), mais aussi avec la scolarité (Absences, retards, notes, …).

Le projet d’ENE de la Région PACA, \correlyce{}, est le premier environnement de ce type à connaître un succès conséquent, c’est pourquoi il est important que le groupe en charge de la Tierce Maintenance Applicative (TMA) de ce projet permette de conserver cette attractivité. Notamment en le rendant robuste, en étendant ses services et en permettant de l’adapter aux environnements des parcs informatiques des différents lycées.

Les régions sont plutôt confiantes dans les projets lancés par le Ministère de l'Éducation comme la relance des Environnement Numériques de Travail (ENT), suite au succès du projet de la région PACA. 

Notre objectif est de confirmer la bonne position de la région PACA dans le domaine des ENE. Il est donc important de continuer sur les bonnes voies tracées par nos prédécesseurs, Pass Tech. 
L’environnement des ENE est fragile et éprouve de la difficulté à être expérimenté par les lycéens et le personnel éducatif, nous attacherons de l’attention à des points qui nous semblent importants comme l’ergonomie du produit, ceci pour améliorer encore le taux d’utilisation de \correlyce{}. Nous serons aussi force de proposition pour donner des idées de services utiles aussi bien pour les élèves que pour les professeurs et la scolarité.

 Nous pensons que \correlyce{} marque le début du succès des ENE en France, et ce sera pour nous un honneur de confirmer ce succès et de lui donner une plus grande ampleur. La région PACA a initié un grand projet et nous souhaitons qu’il serve d’exemple aux autres régions en France et à l’étranger qui souhaiteront mettre en place un ENE adapté. Notre objectif est aussi d’augmenter la visibilité de \correlyce{} et de l’imposer comme référence pour les établissements scolaires.

Il est important de rester en cohérence avec la stratégie de la région. Le recours aux licences libres sera favorisé dans toutes les unités d’oeuvres de la TMA. Les méthodes agiles ayant fait leurs preuves par le passé dans la réalisation de TMA, elles sont pour nous pertinentes dans le cadre de ce projet.
\vfill